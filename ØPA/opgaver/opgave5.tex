\section{Opgave 5}

\begin{frame}
Anders og Helle lever i en lukket økonomi, hvor der kun forbruges fisk og kartofler. Anders kan producere 2kg fisk eller 1kg kartofler, mens Helle kan producere 1kg fisk eller 2kg kartofler. Både Helle og Anders vil kun forbruge de to goder i forholdet 1:1.

Kan de begge stilles bedre ved at handle?
\end{frame}

\begin{frame}{a) Produktionsmulighedsområdet}
\begin{itemize}
  \item Ugentlig produktion (hvis de udelukkende producerer 1 vare)
  \begin{itemize}
    \item Anders: $(F,K)_A = 6\cdot (2,1) = (12,6)$
    \item Helle: $(F,K)_H = 6 \cdot (1,2) = (6, 12)$
  \end{itemize}
\end{itemize}
\begin{figure}
\centering
    \includegraphics[width=0.6\textwidth]{img/comp2}
\end{figure}
Hvad er ligningerne for de to linjer?
\only<2>{\begin{itemize}
  \item Anders: $F_A(K_A) = 12 - 2K_A$
  \item Helle: $F_H(K_H) = 6 - \frac{1}{2}K_H$
\end{itemize}}
\end{frame}


\begin{frame}{b) Forbrug uden handel}
Husk at de begge kun vil forbruge lige dele fisk og kartofler, så for dem begge gælder at $F=K$.

\textbf{Anders:}
\begin{align*}
  F_A &= 12-2K_A \\
  & = 12 - 2F_A \\
  F^*_A &= 4
\end{align*}
Indsæt $F_A = 4$ i den oprindelige ligning
\begin{align*}
  4 &= 12 -2K_H \\
  K^*_H &= 4
\end{align*}
\textbf{Helle:}

Vi kunne gøre som for Anders, men fordi $F_H(K) = F_A^{-1}(K)$ og $F_A^*=K_A^*$ er løsningen $(F_H^*, F_K^*) = (4,4)$.
\end{frame}


\begin{frame}{c) Produktionsmulighedsområde og alternativomkostninger?}

\begin{itemize}
  \item \textbf{Hvad er sammenhængen mellem produktionsmulighedsområdet og alternativomkostningerne?}
  \begin{itemize}
    \item Hældningen på området er en alternativomkostning - \textit{"Hvis Helle øger produktionen af kartofler med 1kg, må hun sænke produktionen af fisk med $\frac{1}{2}$kg".}
  \end{itemize}
  \item \textbf{Hvem har de komparative fordele?}
  \begin{itemize}
    \item Kartofler: Helle. Hun skal kun sænke produktionen af fisk med 1/2 kg for at producere 1kg kartofler ekstra.
    \item Fisk: Anders. Hvorfor?
  \end{itemize}
\end{itemize}

\end{frame}


\begin{frame}{d) Påvirker handel situationen?}

Antag fuld specialisering - så producerer Anders 12kg fisk, og Helle 12kg kartofler. De vil kun forbruge i forholdet 1:1, så de kan bytte 6kg fisk for 6kg kartofler og forbruge $(F,K)=(6,6)$.

\end{frame}

\begin{frame}{d) Påvirker handel situationen?}
\begin{figure}
\centering
    \includegraphics<1>[width=0.9\textwidth]{img/comp2_analysis}
    \includegraphics<2>[width=0.9\textwidth]{img/comp2_analysis_arrow}
\end{figure}

\end{frame}


\begin{frame}{e) Ændring i Helles produktion}
Læg mærke til at ændringen øger helles produktion af både fisk og kartofler forholdsmæssigt lige meget. Derfor er der stadig komparative fordele.

De absolute fordele Anders havde i produktion af fisk er dog forsvundet.
\end{frame}
